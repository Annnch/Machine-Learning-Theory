\documentclass{article}
\usepackage[utf8]{inputenc}

\title{MLT Homework 5}
%\author{Ana Borovac \\ Jonas Haslbeck \\ Bas Haver} % I'll be coming back :-)
\author{Ana Borovac  \\ Bas Haver}

\usepackage{natbib}
\usepackage{graphicx}
\usepackage{subcaption}

\usepackage{amsmath}
\usepackage{amsfonts}
\usepackage{amssymb}
\usepackage{bbm}
\usepackage{mathtools}

\usepackage{url}

\DeclarePairedDelimiter\ceil{\lceil}{\rceil}
\DeclarePairedDelimiter\floor{\lfloor}{\rfloor}

\newcommand{\Hy}{\mathcal{H}}

\newcounter{counterquestion}
\newenvironment{question}[1]
{
\stepcounter{counterquestion}
\section*{Question \thecounterquestion}
\emph{#1} 
} 
{
}

\newcounter{countersubquestion}[counterquestion]
\newenvironment{subquestion}[1]
{
\stepcounter{countersubquestion}
\subsection*{Subquestion \thecounterquestion .\thecountersubquestion}
\emph{#1} 
} 
{
}

\newenvironment{solution}
{
\subsubsection*{Solution}
} 
{
}


\begin{document}

\maketitle

% 1st question
\begin{question}{}

\begin{subquestion}{Consider a hypothesis class $\Hy = \cup_{n = 1}^{\infty} \Hy_n$, where for every $n \in \mathbb{N}$, $\Hy_n$ is finite. Find a weighting function $w: \Hy \to [0, 1]$ such that $\sum_{h \in \Hy} w(h) \leq 1$ and so that for all $h \in \Hy$, $w(h)$ is determined by $|\Hy_{n(h)}|$.}
\begin{solution}
Since we have a countably infinite union of finite sets, we know that the number of elements is countably infinite. Therefore, we can number them as:
\[
h_1, h_2, \dots
\]
If we pick weights as:
\[
w(h_i) = \left(\frac{1}{2^{|H_{n(h_i)}|}}\right)^i; \quad i = 1, 2, \dots
\]
the sum of the weights in the worst case would be, when $|H_{n(h_i)}| = 1$ ($\forall i$):
\[
\sum_{i = 1}^{\infty} w(h_i) = \sum_{i = 1}^{\infty} \left(\frac{1}{2}\right)^i = 1
\]
\end{solution}
\end{subquestion}

\begin{subquestion}{Define such a function $w$ when for all $n$ $\Hy_n$ is countable (possibly infinite).}
\begin{solution}
Countably infinite union of countable sets is again a countable set, so we can choose the same weighted function as before.
\end{solution}
\end{subquestion}

\end{question}


% 2nd question
\begin{question}{7.5}
\begin{solution}

\end{solution}
\end{question}


% 3rd question
\begin{question}{}
\begin{solution}

\end{solution}
\end{question}


\bibliographystyle{plain}
\bibliography{references}
\end{document}