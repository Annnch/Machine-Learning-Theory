\documentclass{article}
\usepackage[utf8]{inputenc}

\title{MLT Homework 6}
\author{Ana Borovac \\ Jonas Haslbeck \\ Bas Haver} % I'll be coming back :-)
%\author{Ana Borovac  \\ Bas Haver}

\usepackage{natbib}
\usepackage{graphicx}
\usepackage{subcaption}

\usepackage{amsmath}
\usepackage{amsfonts}
\usepackage{amssymb}
\usepackage{bbm}
\usepackage{mathtools}

\usepackage{url}

\DeclarePairedDelimiter\ceil{\lceil}{\rceil}
\DeclarePairedDelimiter\floor{\lfloor}{\rfloor}

\newcommand{\Hy}{\mathcal{H}}
\newcommand{\VC}{\text{VCdim}}

\newcounter{counterquestion}
\newenvironment{question}[1]
{
\stepcounter{counterquestion}
\section*{Question \thecounterquestion}
\emph{#1} 
} 
{
}

\newcounter{countersubquestion}[counterquestion]
\newenvironment{subquestion}[1]
{
\stepcounter{countersubquestion}
\subsection*{Subquestion \thecounterquestion .\thecountersubquestion}
\emph{#1} 
} 
{
}

\newenvironment{solution}
{
\subsubsection*{Solution}
} 
{
}


\begin{document}

\maketitle

Raw 0/1-loss with real-valued predictions:
\[
l'(y, \hat{y}) = \frac{1-\text{sign}(y \cdot \hat{y})}{2}, \quad \text{sign}(u) = 
\begin{cases}
1 & \text{if}\ u \geq 0, \\
-1 & \text{if}\ u < 0
\end{cases}
\]

% 1st question
\begin{question}{Suppose that we have a loss function $l$ based on a raw $l'$ as above that is bounded, i.e. $\sup_{\hat{y} \in \hat{Y}, y \in Y} |l'(y, \hat{y})| < \infty$.}

\begin{subquestion}{Suppose that $\Hy$ consists of just one hypothesis, $\Hy = \{h\}$. Show that for all samples $S$, we have that $R(l \circ \Hy \circ S) = 0$.}
\begin{solution}
Let's recall the definition of Rademacher complexity:
\[
R(l \circ \Hy \circ S) = \frac{1}{m} \mathbb{E}_{(y_1, \dots, y_m) \in \{-1, 1\}^m} \left[ \sup_{h \in \Hy} \sum_{y_i} y_i\,l(h, x_i) \right]
\]
In our case we have:
\[
R(l \circ \Hy \circ S) = \frac{1}{m} \mathbb{E}_{(y_1, \dots, y_m) \in \{-1, 1\}^m} \left[ \sum_{y_i} y_i\,l(h, x_i) \right]
\]
With linearity of expected value:
\[
R(l \circ \Hy \circ S) = \frac{1}{m} \sum_{y_i} l(h, x_i)\,\mathbb{E}_{(y_1, \dots, y_m) \in \{-1, 1\}^m} \left[ y_i \right]
\]  
Let calculate the expected value of $y_i$:
\[
\mathbb{E}[y_i] = -1 \cdot P(y_i = -1) + 1 \cdot P(y_i = 1) = -p + p = 0
\]
where $P(y_i = 1) = P(y_i = -1) = p$. Now we can conclude:
\[
R(l \circ \Hy \circ S) = \frac{1}{m} \sum_{y_i} l(h, x_i) \cdot 0 = 0
\]  
\end{solution}
\end{subquestion}

\begin{subquestion}{Consider a hypothesis class $\Hy$ and some hypothesis $h'$. Show that
\[
R(l \circ \Hy \cup \{h'\} \circ S) \geq R(l \circ \Hy \circ S),
\]
and explain why it follows that Rademacher complexity is always nonnegative.}
\begin{solution}
First we are going to write what is $R(l \circ \Hy \cup \{h'\} \circ S)$:
\[
R(l \circ \Hy \circ S) = \frac{1}{m} \mathbb{E}_{(y_1, \dots, y_m) \in \{-1, 1\}^m} \left[ \sup_{h \in \Hy \cup \{h'\}} \sum_{y_i} y_i\,l(h, x_i) \right]
\]
% https://cstheory.stackexchange.com/questions/27821/are-empirical-rademacher-complexity-always-positive

In the previous subquestion we proved that Rademacher complexity of hypothesis class with one hypothesis equals 0, now we proved that if we add a hypothesis to a hypothesis class we get at least as much as we did before. So, Rademacher complexity is nonnegative.
\end{solution}
\end{subquestion}

\begin{subquestion}{Rademacher complexity is often used as a tool to show that some given hypothesis
class $\Hy$ is PAC-learnable. Show that Theorem 26.5 in the book implies agnostic
PAC-learnability for $\Hy$ whenever we can prove that expected Rademacher complexity
satisfies
\[
\lim_{m \to \infty} \sup_{D} \mathbb{E}_{S \sim D^m} \left[ R(l \circ \Hy \circ S) \right] = 0
\]
where the supremum is over all probability distributions $D$ that can be defined on $Z = X \times Y$.}
\begin{solution}
\end{solution}
\end{subquestion}

\end{question}

\begin{question}{(Lipschitz Losses) Consider a raw loss function $l'$ as above with prediction set $\hat{Y}$; here $\hat{Y}$ can either be $\mathbb{R}$ or a closed and bounded interval in $\mathbb{R}$. We say that $l'$ is $\rho-Lipschitz$ if for all $a, b \in \hat{Y}$ we have
\[
\sup_{y \in Y} |l'(y, a) - l'(y, b)| \leq \rho |a - b|
\]
We say that $l'$ is Lipschitz if it is $\rho$-Lipschitz for some $\rho < \infty$. Intuitively, if a loss function
is Lipschitz then small changes in the predictions imply small changes in the loss. It turns
out that analysis with Rademacher complexity can be done for (raw) loss functions that
are either bounded or Lipschitz. For loss functions that are neither bounded nor Lipschitz,
it is much more problematic. We will now consider Lipschitzness for some of the most
commonly appearing loss functions.}

\begin{subquestion}{Show that the 0/1-loss for real-valued predictions is not Lipschitz.}
\begin{solution}
0/1-loss for real-valued predictions is defined as:
\[
l'(y, \hat{y}) = \frac{1-\text{sign}(y \cdot \hat{y})}{2}, \quad \text{sign}(u) = 
\begin{cases}
1 & \text{if}\ u \geq 0, \\
-1 & \text{if}\ u < 0
\end{cases}
\]
It follows:
\begin{align*}
\sup_{y \in Y} |l'(y, a) - l'(y, b)| & = \sup_{y \in Y} \left| \frac{1 - \text{sign}(y \cdot a)}{2} - \frac{1 - \text{sign}(y \cdot b)}{2} \right| \\
& = \frac{1}{2} \sup_{y \in Y} |- \text{sign}(y \cdot a) + \text{sign}(y \cdot b)| \\
& = \frac{1}{2} \sup_{y \in Y} \left| (a - b) \frac{\text{sign}(y \cdot a) - \text{sign}(y \cdot b)}{a-b} \right| \\
& = \frac{1}{2}|a - b| \sup_{y \in Y} \left| \frac{\text{sign}(y \cdot a) - \text{sign}(y \cdot b)}{a-b} \right| \\
& = \frac{1}{2}|a - b| \frac{ \sup_{y \in Y} |\text{sign}(y \cdot a) - \text{sign}(y \cdot b)|}{|a-b|} \\
\end{align*}
If 0/1-loos would be Lipschitz, we would be able to bound:
\[
\frac{ \sup_{y \in Y} |\text{sign}(y \cdot a) - \text{sign}(y \cdot b)|}{2 |a-b|}
\]
We know that the value of numerator is between 0 and 2. If $a$ and $b$ are close together, the whole fraction becomes very large. From that we can conclude that 0/1-loss is not Lipschitz.
\end{solution}
\end{subquestion}

\begin{subquestion}{Is the absolute loss on a bounded domain $Y = \hat{Y} = [-1, 1]$ Lipschitz?}
\begin{solution}
The absolute loss is defined as:
\[
l'(y, \hat{y}) = |y - \hat{y}|
\]
It follows:
\begin{align*}
\sup_{y \in Y} |l'(y, a) - l'(y, b)| & = \sup_{y \in Y} ||y - a| - |y - b|| 
\end{align*}
If $a = b$ we get $\sup_{y \in Y} |l'(y, a) - l'(y, b)| = 0$, so it is bounded with $\rho |a - b|$.

Without loss of generality we can say $a < b$. Now, we have three possible situations:
\begin{itemize}
	\item $y \leq a < b$: If follows $(y - a) \leq 0$ and $(y - b) < 0$, so:
	\begin{align*}
	\sup_{y \in Y} ||y - a| - |y - b|| = \sup_{y \in Y} |-y + a + y - b| \leq \underbrace{1}_{\rho} \cdot |a - b| 
	\end{align*}
	\item $a < y \leq b$: If follows $(y - a) > 0$ and $(y - b) \leq 0$, so:
	\begin{align*}
	\sup_{y \in Y} ||y - a| - |y - b|| & = \sup_{y \in Y} |y - a + y - b| \\
	& = \sup_{y \in Y} \left| (a - b) \left( \frac{y - a}{a - b} + \frac{y - b}{a - b} \right) \right| \\
	& = |a - b| \sup_{y \in Y} \left| \left( \frac{y - a}{a - b} + \frac{y - b}{a - b} \right) \right| \\
	& \leq |a - b| \sup_{y \in Y} \left( \left| \frac{y - a}{a - b} \right| + \left| \frac{y - b}{a - b} \right| \right) \\
	& = |a - b| \sup_{y \in Y} \left( \frac{|y - a|}{|a - b|} + \frac{|y - b|}{|a - b|} \right) \\
	\end{align*}
	From $a < y \leq b$ we know:
	\begin{align*}
	|y - a| & < |a - b|  \quad \Rightarrow \quad \frac{|y - a|}{|a - b|}  < 1 \\
	|y - b| & \leq |a - b|  \quad \Rightarrow \quad \frac{|y - b|}{|a - b|}  \leq 1
	\end{align*}
	So:
	\begin{align*}
	\sup_{y \in Y} ||y - a| - |y - b|| & = |a - b| \sup_{y \in Y} \left( \frac{|y - a|}{|a - b|} + \frac{|y - b|}{|a - b|} \right) \\
	& \leq |a-b| \left( \sup_{y \in Y}\frac{|y - a|}{|a - b|} + \sup_{y \in Y}\frac{|y - b|}{|a - b|} \right) \\
	& \leq |a - b| (1 + 1)  \\
	& = \underbrace{2}_{\rho} \cdot |a - b| 
	\end{align*}
	\item $a < b < y$: If follows $(y - a) > 0$ and $(y - b) > 0$, so:
	\[
	\sup_{y \in Y} ||y - a| - |y - b|| = \sup_{y \in Y} |y - a - y + b| \leq \underbrace{1}_{\rho} \cdot |a - b| 
	\]
\end{itemize} 
To sum up, we can conclude that:
\[
\sup_{y \in Y} |l'(y, a) - l'(y, b)| \leq \underbrace{2}_{\rho} \cdot |a - b| 
\]
So, yes, the absolute loss is Lipschitz.
\end{solution}
\end{subquestion}

\begin{subquestion}{Is the unbounded absolute loss on unbounded domain $Y = \hat{Y} = \mathbb{R}$ Lipschitz?}
\begin{solution}
Yes, we can repeat the proof from bounded domain.
\end{solution}
\end{subquestion}

\begin{subquestion}{Is the squared error loss $l'(y, \hat{y}) = (y - \hat{y})^2$, defined on bounded domain $Y = \hat{Y} = [-1, 1]$ Lipschitz?}
\begin{solution}
From the below:
\begin{align*}
\sup_{y \in Y} |l'(y, a) - l'(y, b)| & = \sup_{y \in Y} |(y - a)^2 - (y - b)^2| = \\
& = \sup_{y \in Y} |y^2 - 2ay + a^2 - y^2 + 2by - b^2| = \\
& = \sup_{y \in Y} |a^2 - b^2 - 2y (a - b)| = \\
& = \sup_{y \in Y} |(a-b)(a+b) - 2y (a - b)| = \\
& = \sup_{y \in Y} |(a-b)(a + b - 2y)| = \\
& = \sup_{y \in Y} |a-b||a + b - 2y| = \\
& = |a-b| \sup_{y \in Y} |a + b - 2y| \leq \\
& \leq |a-b| \sup_{y \in Y} (|a + b| + |2y|) \leq \\
& \leq |a-b| (|a+b| + \sup_{y \in Y} |2y|) \leq \\
& \leq |a-b| (2 + 2) = \\
& = \underbrace{4}_{\rho} \cdot |a - b| 
\end{align*}
it follows that the squared error loss defined on bounded domain is Lipschitz.
\end{solution}
\end{subquestion}

\begin{subquestion}{ Is the unbounded squared error loss defined on unbounded domain $Y = \hat{Y} = \mathbb{R}$ Lipschitz?}
\begin{solution}
No. If it was, we should be able to bound with a constant $< \infty$:
\[
\sup_{y \in Y} |a + b - 2y|
\]
which we can no, since $Y$ is not bounded.
\end{solution}
\end{subquestion}

\end{question}

\bibliographystyle{plain}
\bibliography{references}
\end{document}



















