\documentclass{article}
\usepackage[utf8]{inputenc}

\title{MLT Homework 11}
% \author{Ana Borovac \\ Jonas Haslbeck \\ Bas Haver} % I'll be coming back :-)
\author{Ana Borovac  \\ Bas Haver}

\usepackage{natbib}
\usepackage{graphicx}
\usepackage{subcaption}

\usepackage{amsmath}
\usepackage{amsfonts}
\usepackage{amssymb}
\usepackage{bbm}
\usepackage{mathtools}

\usepackage{url}

\DeclarePairedDelimiter\ceil{\lceil}{\rceil}
\DeclarePairedDelimiter\floor{\lfloor}{\rfloor}

\newcommand{\Hy}{\mathcal{H}}
\newcommand{\VC}{\text{VCdim}}

\newcounter{counterquestion}
\newenvironment{question}[1]
{
\stepcounter{counterquestion}
\section*{Question \thecounterquestion}
\emph{#1} 
} 
{
}

\newcounter{countersubquestion}[counterquestion]
\newenvironment{subquestion}[1]
{
\stepcounter{countersubquestion}
\subsection*{Subquestion \thecounterquestion .\thecountersubquestion}
\emph{#1} 
} 
{
}

\newenvironment{solution}
{
\subsubsection*{Solution}
} 
{
}


\begin{document}

\maketitle

% 1st question
\begin{question}{\textbf{Online Newton Step for large dimension} \\ We saw that $GD$ has a $GD\sqrt{T}$ regret bound for convex loss functions, while ONS has a regret bound of order $d$ ln $T$ for exp-concave loss functions. So can it be that $GD$ is preferable for exp-concave losses in large dimensions $d > \sqrt{T}$? In this exercise you will show that this is not possible with appropriate tuning. Starting from the intermediate ONS regret bound
\[
R_T \leq \frac{\gamma}{2} \epsilon D^2 + \frac{d}{2 \gamma} \ln \left( 1 + \frac{TG^2}{\epsilon d} \right)
\]
prove that carefully tuned ONS also guarantees
\[
R_T \leq GD\sqrt{T}
\]}
\begin{solution}
In the proof we used:
\begin{itemize}
\item $\ln (1 + x) \leq x$
\item $d > \sqrt{T} \quad \Rightarrow \quad \frac{\sqrt{T}}{d} < 1 \quad \Rightarrow \quad \frac{T}{d} < \sqrt{T}$
\item $\epsilon = \frac{G \sqrt{T}}{\gamma D}$
\end{itemize}
It follows:
\begin{align*}
R_T & \leq \frac{\gamma}{2} \epsilon D^2 + \frac{d}{2 \gamma} \ln \left( 1 + \frac{TG^2}{\epsilon d} \right) \\
& \leq \frac{\gamma}{2} \epsilon D^2 + \frac{d}{2 \gamma} \cdot \frac{TG^2}{\epsilon d} \\
& \leq \frac{\gamma}{2} \epsilon D^2 + \frac{\sqrt{T}}{2 \gamma} \cdot \frac{TG^2}{\epsilon d} \\
& \leq \frac{\gamma}{2} \cdot \frac{G \sqrt{T}}{\gamma D} D^2 + \frac{\sqrt{T}}{2 \gamma} \cdot \frac{TG^2}{\frac{G \sqrt{T}}{\gamma D} d} \\
& \leq \frac{1}{2} \cdot \frac{G \sqrt{T}}{1} D + \frac{1}{2} \cdot \frac{TG}{\frac{1}{D} d} \\
& \leq \frac{1}{2} \cdot GD\sqrt{T} + \frac{1}{2} \cdot \frac{T}{d} GD \\
& \leq \frac{1}{2} \cdot GD\sqrt{T} + \frac{1}{2} \cdot \sqrt{T} GD \\
& \leq GD\sqrt{T} \\
\end{align*}
\end{solution}
\end{question}

% 2nd question
\begin{question}{\textbf{ONS as Exponential Weights} \\ In this exercise we recover Online Newton Step as an instance of exponential weights. For simplicity, here we consider ONS without projections (the result is also true with projections). Recall that exp-concave functions $f_t$ satisfy
\[
f_t(u) \geq f_t(x_t) + \langle u - x_t, \nabla_t \rangle + \frac{\gamma}{2} \langle u - x_t, \nabla_t \rangle ^2.
\]
Now consider multivariate Gaussian prior $p_1(u) = N(0, \epsilon^{-1} I)$ and exponential weights update with learning rate $\frac{1}{\gamma}$ applied to the above quadratic lower bound:
\[
p_{t + 1}(u) \propto p_t(u) e^{\frac{1}{\gamma} (f_t(x_t) + \langle u - x_t, \nabla_t \rangle + \frac{\gamma}{2} \langle u - x_t, \nabla_t \rangle ^2)}.
\]
Show that the weights remain multivariate Gaussian, i.e. $p_t = N (\mu_t, \sum_t)$. Show that $\mu_t = x_t$ and $\sum_t = A_{t - 1}^{-1}$.}
\end{question}

% 3rd question
\begin{question}{Determine whether the following functions are exp-concave, and to what degree.}

\begin{subquestion}{$x \mapsto (x - y)^2$ on $x, y \in [-Y, Y]$ First consider a fixed $y$ and then determine the worst-case over all $y \in [-Y, Y]$.}
\begin{solution}
Let's recall the definition of $\alpha$-exp-concave function: A function $f$ is $\alpha$-exp-concave if $x \mapsto e^{-\alpha f(x)}$ is concave.

In our case we get:
\begin{align*}
x \mapsto e^{-\alpha (x - y)^2} \quad & \Rightarrow \quad g(x) = e^{-\alpha (x - y)^2} \\
& \Rightarrow \quad g'(x) = -2 \alpha (x - y)e^{-\alpha (x - y)^2} \\
& \Rightarrow \quad g''(x) = -2 \alpha e^{-\alpha (x - y)^2} + 4 \alpha^2 (x - y)^2 e^{-\alpha (x - y)^2} \\
\end{align*}
%
If $g$ is concave then $g''(x) \leq 0$ for $\forall x$.
\begin{align*}
-2 \alpha e^{-\alpha (x - y)^2} + 4 \alpha^2 (x - y)^2 e^{-\alpha (x - y)^2} & \leq 0 \\
\underbrace{e^{-\alpha (x - y)^2}}_{\geq 0} (-2 \alpha + 4 \alpha^2 (x - y)^2) & \leq 0 \\
2  \alpha ( - 1+ 2 \alpha (x - y)^2) & \leq 0 \\
 - 1+ 2 \alpha (x - y)^2 & \leq 0 \\
\alpha & \leq \frac{1}{2 (x - y)^2} \\
\end{align*}

If we want to inequality to hold for $\forall x$ we need to observe when the fraction on the right is the smallest. So, when does $(x - y)^2$ reach maximum? It depands on $y$, but it is obvious that it is reached when $x$ is $Y$ or $-Y$, so:
\[
\alpha = \frac{1}{2 \max \{ (Y - y)^2, (-Y - y)^2 \}}
\]

In the worst case the difference between $x$ and $y$ is $2Y$. From that it follows:
\[
\alpha = \frac{1}{8 Y^2}
\]
\end{solution}
\end{subquestion}

\begin{subquestion}{$x \mapsto |x|^p$ on $x \in [-1, 1]$ for $p >1$.}
\begin{solution}
\begin{align*}
g(x) = e^{-\alpha |x|^p} 
\end{align*}
Since, $g$ is not differentiable for $x = 0$, we are going analyse concavity on two intervals: $[-1, 0)$ and $(0, 1]$.
\begin{itemize}
\item $x \in [-1, 0)$
\begin{align*}
g(x) & = e^{-\alpha (-x)^p} \\
g'(x) & = \alpha p (-x)^{p - 1} e^{-\alpha (-x)^p}  \\
g''(x) & = - \alpha p (p - 1) (-x)^{p - 2} e^{-\alpha (-x)^p} + \alpha p (-x)^{p - 1} \alpha p (-x)^{p - 1} e^{-\alpha (-x)^p} \\
\end{align*}
$\Rightarrow$
\begin{align*}
g''(x) & \leq 0 \\
- \alpha p (p - 1) (-x)^{p - 2} e^{-\alpha (-x)^p} + \alpha p (-x)^{p - 1} \alpha p (-x)^{p - 1} e^{-\alpha (-x)^p} & \leq 0 \\
- (p - 1) (-x)^{-1} +  (-x)^{p - 1} \alpha p & \leq 0 \\
(-x)^{p - 1} \alpha p & \leq \frac{p - 1}{-x}  \\
\alpha & \leq \frac{p - 1}{p (-x) (-x)^{p - 1} }  \\
\alpha & \leq \frac{p - 1}{p (-x)^{p} }  \\
\end{align*}
$\Rightarrow$
\begin{align*}
\alpha & = \min_{x \in [-1, 0)} \frac{p - 1}{p (-x)^{p} }  \\
\alpha & = \frac{p - 1}{p}
\end{align*}
\item $x \in (0, 1]$
\begin{align*}
g(x) & = e^{-\alpha x^p} \\
g'(x) & = -\alpha p x^{p - 1} e^{-\alpha x^p}  \\
g''(x) & = - \alpha p (p - 1) x^{p - 2} e^{-\alpha x^p} + \alpha p x^{p - 1} \alpha p x^{p - 1} e^{-\alpha x^p} \\
\end{align*}
$\Rightarrow$
\begin{align*}
g''(x) & \leq 0 \\
- \alpha p (p - 1) x^{p - 2} e^{-\alpha x^p} + \alpha p x^{p - 1} \alpha p x^{p - 1} e^{-\alpha x^p} & \leq 0 \\
- (p - 1) x^{-1} + x^{p - 1} \alpha p & \leq 0 \\
x^{p - 1} \alpha p & \leq \frac{p - 1}{x} \\
\alpha & \leq \frac{p - 1}{p x x^{p - 1}} \\
\alpha & \leq \frac{p - 1}{p x^{p}} \\
\end{align*}
$\Rightarrow$
\begin{align*}
\alpha & = \min_{x \in (0, 1]} \frac{p - 1}{p x^{p} }  \\
\alpha & = \frac{p - 1}{p}
\end{align*}
\end{itemize}
We can conclude that $g$ is $\frac{p - 1}{p}$-exp-concave on intervals $[-1, 0)$ and $(0, 1]$. Since $g$ is continuous function on $[-1 ,1]$, it is $\frac{p - 1}{p}$-exp-concave on whole $[-1, 1]$.
\end{solution}
\end{subquestion}

\end{question}

% 4th question
\begin{question}{}

\begin{subquestion}{ Let $f(w) = g( \langle w, x \rangle )$ for some fixed $x$. Show that $f$ is $\alpha$-exp-concave when $g$ is $\alpha$-exp-concave.}
\begin{solution}
\end{solution}
\end{subquestion}

\begin{subquestion}{Let $f(w, y) = g( \langle w, x \rangle, y)$ for some fixed $x$. Show that $f$ is $\eta$-mixable when $g$ is $\eta$-mixable.}
\begin{solution}
\end{solution}
\end{subquestion}

\end{question}

\bibliographystyle{plain}
\bibliography{references}
\end{document}



















