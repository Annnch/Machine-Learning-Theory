\documentclass{article}
\usepackage[utf8]{inputenc}

\title{MLT Homework 4}
\author{Ana Borovac \\ Jonas Haslbeck \\ Bas Haver}

\usepackage{natbib}
\usepackage{graphicx}
\usepackage{subcaption}

\usepackage{amsmath}
\usepackage{amsfonts}
\usepackage{amssymb}
\usepackage{bbm}
\usepackage{mathtools}

\usepackage{url}

\DeclarePairedDelimiter\ceil{\lceil}{\rceil}
\DeclarePairedDelimiter\floor{\lfloor}{\rfloor}

\newcounter{counterquestion}
\newenvironment{question}[1]
{
\stepcounter{counterquestion}
\section*{Question \thecounterquestion}
\emph{#1} 
} 
{
}

\newcounter{countersubquestion}[counterquestion]
\newenvironment{subquestion}[1]
{
\stepcounter{countersubquestion}
\subsection*{Subquestion \thecounterquestion .\thecountersubquestion}
\emph{#1} 
} 
{
}

\newenvironment{solution}
{
\subsubsection*{Solution}
} 
{
}


\begin{document}

\maketitle

% 1st question
\begin{question}{We have shown that for a finite hypothesis class $\mathcal{H}$, $\text{VCdim}(\mathcal{H}) \leq \floor*{\log(|\mathcal{H}|)}$. However, this is just an upper bound. The VC-dimension of a class can be much lower than that.}
\begin{subquestion}{Find an example of a class $\mathcal{H}$ of functions over the real interval $\mathcal{X} = [0, 1]$ such that $\mathcal{H}$ is infinite while $\text{VCdim}(\mathcal{H}) = 1$.}
\begin{solution}
Let's define hypothesis class as:
\[
\mathcal{H} = \{h_{a}:\ a \in [0, 1]\}
\]
\[
h_{a}(x) = 
\begin{cases}
1; & x \geq a \\
0; & x < a \\
\end{cases} 
\]
%
From definition we know $|\mathcal{H}| = \infty$. Now we need to prove that $\text{VCdim}(\mathcal{H}) = 1$.

\begin{figure}[h!]
    \centering
    \begin{subfigure}[t]{0.4\textwidth}
        \centering
        \includegraphics[width=0.9\textwidth]{Question11.png}
        \caption{If point is labeled “1”.}
    \end{subfigure}
    \begin{subfigure}[t]{0.4\textwidth}
        \centering
        \includegraphics[width=0.9\textwidth]{Question12.png}
        \caption{If point is labeled “0”.}
    \end{subfigure}
    \caption{Proof that $\text{VCdim}(\mathcal{H}) \geq 1$.}
    \label{fig: question111}
\end{figure}

\begin{itemize}
\item $\text{VCdim}(\mathcal{H}) \geq 1$: The proof we can see from the figure \ref{fig: question111}.
\item $\text{VCdim}(\mathcal{H}) \leq 1$: From the figure \ref{fig: question112} it is seen that hypothesis class $\mathcal{H}$ does not shatter a set of two points (no matter how we position them).
\end{itemize}

\begin{figure}[h!]
\centering
\includegraphics[width=0.5\textwidth]{Question13.png}
\caption{The problem we have when trying to shatter a set of two points.}
\label{fig: question112}
\end{figure}
\end{solution}
\end{subquestion}

\begin{subquestion}{Give an example of a finite hypothesis class $\mathcal{H}$ over domain $\mathcal{X} = [0, 1]$, where $\text{VCdim}(\mathcal{H}) = \floor*{\log_2(|\mathcal{H}|)}$}
\begin{solution}
Let's define hypothesis class:
\[
\mathcal{H} = \{h_0, h_1\}
\]
where $h_0(x) = 0$ ($\forall x$) and $h_1(x) = 1$ ($\forall x$). We would like to prove that $\text{VCdim}(\mathcal{H}) =  \floor*{\log_2(|\mathcal{H}|)} =  \floor*{\log_2(2)} = 1$.
\begin{itemize}
\item $\text{VCdim}(\mathcal{H}) \geq 1$: If we want to label a $x \in [0, 1]$ as 1, we pick $h_1$ as hypothesis, otherwise we pick $h_0$. So, $\text{VCdim}(\mathcal{H}) \geq 1$.
\item $\text{VCdim}(\mathcal{H}) \leq 1$: Let say that we have a set of two points. If we want to label one of the point with 1 and the other with 0, there does not exist a hypothesis in hypothesis class which can label two points differently. 
\end{itemize}
We can conclude that $\text{VCdim}(\mathcal{H}) = 1$.
\end{solution}
\end{subquestion}
\end{question}


% 2nd question
\begin{question}{6.8}
\begin{solution}

\end{solution}
\end{question}


% 3rd question
\begin{question}{6.9}
\begin{solution}

\end{solution}
\end{question}


% 4th question
\begin{question}{6.11}
\begin{solution}

\end{solution}
\end{question}

\bibliographystyle{plain}
\bibliography{references}
\end{document}